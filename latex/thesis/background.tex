%
% $Id: chapter2.tex 2612 2008-06-03 18:32:54Z jozinke $
%

\chapter{Background}
\label{sec:background}

The following section introduces the concept of \ac{HIDS} with the specific example of Fail2ban and presents the problem setting 
this thesis aims to solve. In addition to that, an overview over common types of Inter-Process Communication and existing IPC based logging solution is given. 
Finally, external libraries and other software used for the implementation and evaluation of the Proof-of-Concept IPS are introduced.    

\section{Host-based Intrusion Detection / Prevention}
\label{sec:hids}
The idea of specialized software for detecting intrusion attempts and other 
security threads goes as far back as 1980, when James Anderson published a study on 
``Computer security threat monitoring and surveillance''\cite{anderson1980}. In 1987, Dorothy
Denning presented a seminal model for Intrusion Detection Systems, that suggested the use of pattern matching based on
statistical analysis of audit records generated by a system, in order to detect abnormal user behavior \cite{denning1987}.         
Intrusion Detection Systems in general, gather data from a multitude of sources,
which is then processed to identify and report potential threads. Host-based Introduction Detection Systems in particular,
use information that is provided by the hosts under their supervision. This includes event logs of applications, as well as operating system (\ac{OS}) based information,
such as user logins, file system operations or systemcalls. The analysis of the gathered data can be divided into two categories: 1. Misuse based detection relies on predefined 
patterns of misuse or malicious behavior, which are then matched against the observed behavior in the data. 2. Anomaly based detection uses statistical analysis to identify deviations
from the norm, thereby also being able identify attacks, that have not been previously observed \cite{vigna2006}. Intrusion Prevention Systems (\ac{IPS}) constitute a special class of IDS, which
are not only capable of detecting an attack, but also take measure to prevent or mitigate it.

\subsection{Fail2ban}
\label{sec:fail2ban}

Fail2ban is a open source IPS for POSIX Systems, that is widely used to protect web servers, for instance
against brute-force login attempts, as well as other types of attacks \cite{}. To identify potentially malicious clients,
Fail2ban makes use of application logs, that are parsed based on a predefined filter. Fail2ban uses configuration units called `Jails'  
, that allow for the customization to a wide range of applications. A Jail defines the path to the application log, the filter being applied to the log messages within the logfile and an action, that
is executed on client matching the filter criteria. In addition to that, Jails contain further parameters, such as the threshold of matches a client need to reach, in order for the action to be executed,
as well as the duration of the action. The filter component of a Jail defines a set of regular expressions, that are used to identify certain events in a log, like an unsuccessful login attempts or the 
exceeding of a rate limit. the filter also obtains a clients IP address as well as the date and time of the log messages, to determine, if the event occurred in a relevant time frame. 



\section{Inter-Process Communication}
\label{sec:ipc}

\subsection{Types of IPC}
\label{sec:ipc_types}

\subsection{IPC based logging}

Syslog, Rsyslog

\section{Special Software}
\label{sec:softwar}

\subsection{Hyperscan}
\label{sec:hyperscan}

Hyperscan is a open source regular expressions matching engine developed by Intel. 
It is specifically designed for high performance use cases, such as the application in security contexts and is being used by the intrusion detection systems Snort and Suricata.  
The process of regular expressions matching with Hyperscan is separated into compile- and run-time. At compile-time a set regular expressions in string representation are compiled into a 
database, with additional configuration options  

\subsection{io\_uring}
\label{sec:io_uring}

\subsection{Trex}
\label{sec:trex}
