%
% $Id: chapter1.tex 2908 2008-11-19 13:56:30Z sliske $
%

\chapter{Introduction}
\label{sec:intro}

Ever since the advent of the public internet in the past century, bandwidths available to both
private and commercial users have continuously expanded. While this opens up opportunities for new services and bandwidth-extensive 
applications, it also poses new security challenges for dealing with potentially malicious network traffic. In addition to traditional firewalls, 
\acp{IDS} are a commonly used security measure to protect hosts in both public and private networks. They
monitor system and network resources, to identify and respond to security breaches and potential attacks.\acp{HIDS} in particular, rely
on information provided by the host system, one possible source for which are application logs. Traditionally, \ac{HIDS} gather log information by 
parsing the logfiles maintained by applications or the \ac{OS}. One notable example for this, is the open source software Fail2ban. Fail2ban is
an \ac{IPS}, that is widely used to protect servers against different network based attacks. It
utilizes logfiles to identify and subsequently ban clients that fit predefined patterns of harmful behavior. 
\par
Previous work has shown, that the performance of Fail2ban scales poorly, when large amount of log messages need to be processed in a short time frame \cite{mikolajczak2022}. This constitutes
a problem for high bandwidth networks, as millions of clients can send request simultaneously, leading to a significant influx of security relevant log events in case of an attack.  
If these messages are not processed, the system becomes vulnerable to security breaches. One possible reason for Fail2bans performance issues is the high latency associated with
the file-based transmission of log messages. This inhibits its responsiveness in a high-bandwidth attack and consequently leads to Fail2ban being overwhelmed by the amount of log events to processes. 
Therefore, replacing log file parsing with faster methods of log message transmission, could improve the capability to handle the high load scenarios that can occur in modern networks. 
\par
The goal of this thesis will be the design and implementation of a new \ac{IPC} based architecture for the
transmission of log messages, that is able to facilitate low latency communication
between sender and receiver in the context of \ac{HIDS}. Additionally, the design should be able to scale to multiple
recipients, in order to accommodate more complex security system, in which several processes
require access to a hosts application log. For the purpose of evaluation, a \ac{PoC} \ac{IPS} will be 
developed, that utilizes the proposed \ac{IPC} architecture to receive log messages and ban malicious
clients in the style of Fail2ban. The \ac{PoC} will be evaluated and compared to Fail2ban, to asses the viability of the new architecture. 
\par 
This thesis will be structured as follows: The background section provides 
information on relevant concepts and introduces the problem setting in further detail. Subsequently, design and implementation
of the new \ac{IPC} architecture and the proof of concept \ac{IPS} are presented. The evaluation section provides the experimental design and the results for the 
evaluation of both Fail2ban and the \ac{PoC}. The final section concludes and offers an outlook for further development.   