%
% chapter4.tex
%

\chapter{Implementation}
\label{cha:implementation}
In this chapter, the implementation of the previously discussed design (Reference\@: \ref{sec:Design_and_abstractions}) for the unix domain socket architecture is presented.
This includes an explanation of the write and read API.
All error codes are numeric and defined the file \texttt{io\_ipc.h}

\section{Auxiliary functions and structures}
When utilizing the socket IPC type, some shared resources seen in \ref{alg:variables:shared} need to be set up.
None of these are modified during runtime.

\begin{algorithm}[h!]
    \lstinputlisting[language=c, firstline=18, lastline=22]{code/sock_comm.h}
    \caption[Socket: Shared variables]{Parameters shared between readers and writers.}
    \label{alg:variables:shared}
\end{algorithm}

To ensure that no writer gets stuck continually checking for new sockets, the global variable \texttt{MAX\_AMOUNT\_OF\_SOCKETS} is defined.
This also acts as an upper limit in regards to the amount of reader processes that can attach to the unix domain socket IPC.
The unix domain sockets were bound to the filesystem, resulting in a common path to the location of all sockets needing to be supplied to both readers and writers.
However, this \texttt{SOCKET\_NAME\_TEMPLATE} is not the full path to each socket.
During runtime, each reader process trying to attach will append this name template with their own reader ID.
The reader ID is determined by claiming the first ID not already in use.
Since the length of the reader ID being appended to the \texttt{SOCKET\_NAME\_TEMPLATE} can vary, a maximum length for this template is defined in \texttt{SOCKET\_TEMPLATE\_LENGTH}.

Separating functions utilized by both readers and writers results in an unwieldy API.
Shared usage of functions by both sides is achieved by supplying function calls with the role of the calling process, either \texttt{SOCK\_WRITER} or \texttt{SOCK\_READER}.

\begin{algorithm}[h!]
    \lstinputlisting[language=c, firstline=88, lastline=89]{code/sock_comm.h}
    \caption[Socket: Socket initialization]{Initialization function for both reader and writer processes.}
    \label{alg:sock:init}
\end{algorithm}

Therefore the function initializing communication between processes, \texttt{sock\_init} displayed in \ref{alg:sock:init} only requires a structure of parameters and the role of the calling process.
Defining a union containing both writer and reader structures as seen in \ref{alg:sock:union} allows the user of the API to provide either one as a parameter for the same function.
The actual purpose of \texttt{sock\_init} is to enable connection between writer and readers.
Writers are provided with a list of possible locations of unix domain sockets belonging to reader processes.
Meanwhile, readers are assigned a path, in which they create a unix domain socket.
All sockets are set to be of the type SOCK\_SEQPACKET.

\begin{algorithm}[h!]
    \lstinputlisting[language=c, firstline=75, lastline=78]{code/sock_comm.h}
    \caption[Socket: Union for flexible function calling]{Union containing either the parameters of a writer or reader process.}
    \label{alg:sock:union}
\end{algorithm}

While other IPC types such as shared memory required an orderly detachment of writers and readers, this is not necessary for the socket approach.
Instead, when terminating a reader process, only closure of the corresponding unix domain socket is necessary.
Stopping a writer process currently results in deconstructing the entire unix domain socket architecture.
This results in the functions \texttt{socket\_finalize} and \texttt{socket\_cleanup}, as shown in \ref{alg:sock:finalize} and \ref{alg:sock:cleanup} respectively, being identical in behavior.
In fact, \texttt{socket\_finalize} simply calls \texttt{socket\_cleanup} and was only provided in the socket API to make a seamless replacement of other finalize-style functions when switching IPC types possible.

\begin{algorithm}[h!]
    \lstinputlisting[language=c, firstline=125, lastline=125]{code/sock_comm.h}
    \caption[Socket: Socket finalization]{Initializes cleanup of socket IPC.}
    \label{alg:sock:finalize}
\end{algorithm}

\begin{algorithm}[h!]
    \lstinputlisting[language=c, firstline=135, lastline=135]{code/sock_comm.h}
    \caption[Socket: Socket cleanup]{Cleanup of socket IPC.}
    \label{alg:sock:cleanup}
\end{algorithm}

\section{Write API}
The write API consists out of a single, versatile function \texttt{sock\_writev}.
See \ref{alg_sock} for its definition.

It requires four arguments:
\begin{itemize}
    \item A pointer to an instance of the structure \texttt{sock\_writer\_arg\_t} which will be introduced shortly.
    \item A pointer to an array of \texttt{iovec} structures.
            Each \texttt{iovec} structure defines a separate memory region of variable size, acting as a buffer.
            An entire array of such structures can be represents a vector of memory regions\cite{man:iovec}.
    \item The integer \texttt{invalid\_count} represents the number of log messages located in the iovec array.
    \item Finally, the maximum number of receiving sockets is given via the parameter \texttt{maxNumOfSocks}.
\end{itemize}

\begin{algorithm}[h!]
    \lstinputlisting[language=c, firstline=101, lastline=103]{code/sock_comm.h}
    \caption[Socket: Write API]{Write API for the unix domain socket architecture}
    \label{alg:sock}
\end{algorithm}

When calling \texttt{sock\_writev} the first thing being performed is a check for available unix domain sockets.
This can be considered analogous to checking for new reader because of the strict one-to-one mapping between sockets and readers.
If new sockets were identified, a connection with that socket is established and saved for future calls of this function.
Then, all sockets with an existing connection are sent the data located in the iovec structures using the function \texttt{write}.
On success, the function returns the number of sent messages via the socket IPC.

\section{Read API}
Read API

\section{Features}
Features