%
% chapter6.tex
%

\chapter{Conclusion \& Outlook}
In this thesis, the previously developed light-weight \ac{IPS} Simplefail2ban and its selection of \ac{IPC} was expanded to include UNIX domain sockets.
With it being the first kernel-based \ac{IPC} available, thorough measurements were conducted to evaluate its performance to the already implemented shared memory and file-base \ac{IPC} modes.
Expectations were that UNIX domain sockets would not outperform shared memory, because of a constant need for context-switches between kernel- and userspace.
This initial hypothesis turned out to be true.
The socket \ac{IPC} type was beat in all analyzed metrics (number of unwanted requests dropped, number of log messages sent and \ac{CPU} time) by the shared memory mode of Simplefail2ban.
However, performance of the UNIX domain socket mode was still well up to the task.
It remained competitive in experiments utilizing only one process to receive data, performing only around 2 percent worse than its shared memory counterpart.
During less intense traffic flow, this small disadvantage in performance shrunk even further.
Data indicates that latency in the socket \ac{IPC} type was at least on par with the shared memory mode.
Rather, a lack of bandwidth and increased drain on system resources are the main causes identified for the observed decrease in performance.
Conversely, UNIX domain socket did consistently outperform the file-based \ac{IPC}, albeit by a small fraction: regularly being less than one percentage point in relative drop rate.
Overall, the socket \ac{IPC} type was always able to block over 95,5 percent (and 98,5 percent on average) of all incoming traffic, therefore defending against \ac{DoS} attacks successfully.

While never being an explicitly desired feature, the socket \ac{IPC} type does provide the option to attach, up to a pre-defined maximum, and detach a varying number of both writer and reader processes during runtime.
In contrast, the shared memory \ac{IPC} only provides the option to attach multiple reader processes.
This allows for, in theory, flexible reusing of the socket \ac{IPC} architecture for other applications.
Regrettably however, usage of UNIX domain sockets in scenarios with multiple reader processes is not recommended due to a lack of bandwidth, resulting in the socket \ac{IPC} performing significantly worse than the shared memory \ac{IPC}.
At a rate of 20m invalid \ac{PPS}, defending against a \ac{DoS} attack in a single ban cycle was unfeasible when employing UNIX domain sockets.
Yet, after multiple ban cycles, Simplefail2ban was able to recover and repel the incoming \ac{DoS} attack.
With 30m invalid \ac{PPS}, a recovery became impossible.
The shared memory \ac{IPC} type was able to keep its performance up even when supplying a second reader process, and disabling the overwrite feature.

Potential improvements of the socket \ac{IPC} should focus on increasing the bandwidth.
This makes it possible to react more efficiently and effectively in the event of a sudden influx of messages, primarily occurring at the beginning of a ban cycle or \ac{DoS} attack.

Overall, this thesis proved that the kernel-based UNIX domain sockets remain a viable option as \ac{IPC} coming reasonably close to, yet unable to surpass the higher performance of the shared memory \ac{IPC} type.
Continued development could focus on the possibility of scaling the socket \ac{IPC} beyond the local system by employing internet sockets.
Providing an improved high-level \ac{API} for easier integrability with established real-world applications, such as syslog or journald, is also worthwhile investigating.